{\footnotesize{\bf Ferdinand von Ramm} föddes 1890 i Tavastehus. Han tillhörde en baltisk 
adelssläkt och blev student från Ritter- und Domschule i Reval 1911.
När han i mitten av oktober sammahöst inledde sina studier vid 
Helsingfors universitet valde han att skriva in sig i Vasa nation, 
där han hade släktingar på mödernet, och alltsedan dess förblev han 
vasung och österbottning till själ och hjärta - trots att han knappast 
tillbringade mer än åren under vinter- och fortsättningskriget i 
Österbotten. Rammen berättade själv så här:

``Det var den 9 november 1911 som jag begav mig till
Porthansfesten, iklädd frack och prydd med det röd-gul-svarta
bandet. Mottagandet jag fick var överväldigande vänligt. Man
må ta i betraktande, att jag för herrarna var en helt främmande
figur. Den dåtida inspektorn, prof. Otto Engström, hälsade mig
hjärtligt välkommen, och den kvällen kommer evigt att leva kvar
i mitt minne. Där lärde jag känna kamrater även av finsk
nationalitet, med vilka en trogen vänskap bundit mig genom åren.
Nu började för mig det glada studentlivet inom min egen
korporation. Höstterminen 1911 gick emellertid mot sitt slut.
Terminsavslutningens höjdpunkt bildade julfesten, 'Lilla jul'.
Fastän jag bara var gulnäbb tilldelades jag den stora äran att bli
vald i den kommitté, som fått sig festens arrangemang
anförtrodda, inhandlande av lustiga små presenter m.m. Denna
första julfest i kretsen av mina kamrater kommer alltid att vara
mig oförglömlig. Wasa nation hade på den tiden inget eget 'hem',
utan samlades i Restaurant C. F. Nyberg, belägen mitt emot
'senatsbyggnaden' Som lokal var detta etablissemang egentligen
ganska primitivt, men maten varförträfflig och betjäningen prima.
Många glada sånger sjöngs denna oförglömtiga kväll och först
sent på natten skiljdes man med den glada förhoppningen om
ett återseende under den kommande vårterminen 1912. Var
och en strävade nu hemåt. Ja, jag må säga det var gyllene
tider. Sorglös och glad.''

Ferdinand von Ramm var så genomsyrad av tysk
korporationsanda med dess speciella disciplins- och
hedersbegrepp att han räknade det som sin själfallna plikt att
i början av varje termin personligen anmäla sig hos nationens
kurator och att aldrig försumma ett tillfälle då nationen samlades.
I den mer demokratiska nationen var han en sällsam främmande
fågel med sin spinkiga, något degenererade typ och sin säregna
blandning av gammal fin herrgårdskultur och överdådigt
studentikost festhumör. I hela studentvärlden var han känd och
omåttligt populär, och ännu långt efter det han lämnat studieåren
bakom sig hälsades han med jubelrop var gång han uppenbarade
sig i ett glatt lag på Ostrobotnia. Någon examen avlade han
aldrig, trots grundliga - speciellt historiska - kunskaper och ett
fenomenalt minne. Efter att revolutionen gjort slut på familjens
förmögenhet försörjde han i fortsättningen sig och sin mor på
en enkel kontorstjänst som utrikeskorrespondent hos
andelslaget Labor. Han behärskade sex olika spåk, men inte
finska. Hans ryska var perfekt vilket kom till stor nytta på hans
arbetsplats. Efter kriget sjukpensionerades han och gifte sig
hösten 1947, till sina gamla vänners stora förvåning.

Men ännu när han stod i beråd att med sin hustru bosätta sig i
hennes villa vid Medelhavet för sin återstående livstid,
försummade han inte att komma upp till de äldres samkväm på
Ostrobotnia för att ta farväl av sin nation. Med sviktande
själskrafter erinrade han sig sin ungdoms dårskaper och nertecknade 
dem i en på tyska avfattad relation, som han år
1952 översände till Vasa nation för att förvaras i dess arkiv.
Ferdinand von Ramm avled i Frankrike den 7 december 1960.

Rammens visa skrevs vårvintern 1913 av Thure Svedlin,
nationens kurator 1918-20, till ett spex. Nationens förste kurator
Artur Eklund berättar i dikten ``Ett Königs- minne'', hur han, Svedlin
och Alfred Fahler samlats i det legendäriska ``Gröna rummet'' på
restaurang Karl König vid Mikaelsgatan för att med gemensamma
krafter åstadkomma det spex som skulle bli huvudnumret på ett
allmänt svenskt studentsamkväm. Restaurangens populära bar
kunde under 1910-talet räkna Ferdinand von Ramm bland sina
trognaste stamkunder, vilket förklarar hur han och hans trogne
kumpan Harald ``Mosse'' Monsen, från Nylands nation, dyker
upp för att lyssna till den nyskrivna sången. Sången har länge
funnits med i nationens sångbok, men melodin glömdes under
slutet av sextiotalet, men på initiativ av Sture Björk, en av
nationens mer legendariska sångledare, togs den upp igen i
samband med nationens 75-årsjubileum.}
\sclearpage



