

\beginsong{Nikodemos}[
	by={Lars Huldén}] 

\beginverse*
- bibelvisa skriven för Vasa nation 29.2.2008 vid kyrkkaffer efter John Vikströms predikan.
\endverse

\beginverse*
Nikodemos, Nikodemos
var en höglärd man.
Han gick mitt i natten
för att söka katten.
Hem till någon, hem till någon
gick istället han.
\endverse

\beginverse*
När han bulta', när han bulta' 
steg den andre opp.
Knappt han öppnat dörren
åt främlingen förrän
frågor hagla', frågor hagla' 
mot hans huvudknopp.
\endverse

\beginverse*
"Alla vet att, alla vet att
du är pedagog,
en gudabenådad
hos oss aldrig skådad.
Men det andra, men det andra
är väl bara båg?"
\endverse

\beginverse*
Nikodemos, Nikodemos
ville veta mer
om den nya läran,
kom med en begäran:
"Gör ett under, gör ett under!
Tror gör den som ser."
\endverse

\beginverse*
Men i stället, men i stället
fick han många ord.
Gubben blev besviken;
av metaforiken
kände han sig, kände han sig
platt tillintetgjord,
\endverse

\beginverse*
"Inte kan man, inte kan man
nånsin födas om.
Det förstår väl fler att
är för komplicerat",
mena' gubben, mena' gubben,
ångra' att han kom.
\endverse

\beginverse*
Vinden viner, vinden viner
och far dit den far.
"Vad är det med det då?
Alla vet det är så!"
Ingivelsens, ingivelsens
virvelvind det var.
\endverse

\beginverse*
Full förklaring, full förklaring
är det svårt att få.
Såvitt man förstår är
allting metaforer,
som får tydas, som får tydas
Det är vackert så.
\endverse
\endsong
