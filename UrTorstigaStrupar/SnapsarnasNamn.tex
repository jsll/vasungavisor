\begin{intersong}

\textbf{SNAPSARNAS NAMN}

\begin{small}
Alltsedan den blida tid då vårt brännvin uppfanns, har umgänget med detsamma varit omgärdat av en stängeligen upprätthållen ritual. Vad vi vet därom förtäljer traditionen. Brännvinet inmundigades i bestämda kvantiteter benämnda "Snapsen" eller "Supen". Enär dessa av goda skäl inte kunde tillåtas intagas godtyckligen, fick de snart alltefter sina rituella ordningsnummer specifika namn, av vilka vi idag känna tretton. Dessa är Helan, Halfvan, Tersen, Quarten, Quinten, Rifvan, Rafflan, Rännan, Smuttan, Smuttans unge, Femton droppar, Lilla Manasse och Lilla Manasses broder. Detta kulturarv bjuder oss att följa denna ritual och inte tillåta stundens nycker nagga dryckesdisciplinen i kanten. Vad skulle våra dryckesfäder känna i sina hjärtan om de från sina molnkanter nedblickade på en snapsritual innehållande "Langbeinakrank" istället för Rifvan och "Tjorven" eller, hemska tanke "Sjunkbomben" istället för den ärevördiga Lilla Manassen? Ruelse och vemod! Sentida studenter plägnar dock att utöver de traditionella alltefter tillfället tillägga snapsarna Kreaturets återuppståndelse och Inspektors klagan. Detta kunde emellertid klandras med avseende på måttlighet i denna värld befolkad av ett i fysisk måtto något förvekligat släkte, men som bekant tillhör inte omåttlighet dödssynderna, ej heller är måttlighet den högsta av dygder. Dock torde detta gruvligen utökade antalet snapsar anses vara ett förvekligat släkte tillfyllest. Efterföljande samling gör inte anspråk på uttömlighet ifråga om de olika smakriktningarna inom supvisornas genre, från tradition till vomeringsromantik, men likförbannat bör de uppstämmas med samma kraft som fordom innan snapsen under andaktsfull tystnad intagas. I kvistiga fall angående melodi eller dryckesdisciplin torde man vända sig till någon äldre vasung, lämpligen sångledaren (vasungen med sångvärjan; akta dig för den, du).
\end{small}

\end{intersong}

