\beginsong{O alte Burschenherrlichkeit}[
	index={Rückblick eines alten Burschen},
	index={O, jerum, jerum, jerum}]

\beginverse* 
O alte Burschenherrlichkeit,
wohin bist du entschwunden?
Nie kehrst du wieder, goldne Zeit,
so froh und ungebunden!
Vergebens spähe ich umher,
ich finde deine Spur nicht mehr,
o jerum, jerum, jerum, o quae mutatio rerum.
\endverse

\beginverse* 
Den Burschenhut bedeckt der Staub,
es sank der Flaus in Trümmer,
der Schläger ward des Rostes Raub,
erblichen ist sein Schimmer,
verklungen der Kommersgesang,
verhallt Rapier- und Sporenklang,
o jerum, o quae mutatio rerum.
\endverse

\beginverse* 
Wo sind sie, die vom breiten Stein
nicht wankten und nicht wichen,
die ohne Moos bei Scherz und Wein
den Herr'n der Erde glichen?
Sie zogen mit gesenktem Blick
in das Philisterland zurück,
o jerum, o quae mutatio rerum.
\endverse

\beginverse* 
Da schreibt mit finsterm Amtsgesicht
der eine Relationen,
der andre seufzt beim Unterricht,
und der macht Rezensionen,
der schilt die sünd'ge Seele aus
und der flickt ihr verfall'nes Haus,
o jerum, o quae mutatio rerum.
\endverse

\beginverse* 
Allein das rechte Burschenherz
kann nimmermehr erkalten;
im Ernste wird, wie hier im Scherz,
der rechte Sinn stets walten;
die alte Schale nur ist fern,
geblieben ist uns doch der Kern,
und den laßt fest uns halten.
\endverse

\beginverse*
Drum, Freunde, reichet euch die Hand,
damit es sich erneure,
der alten Freundschaft heil'ges Band,
das alte Band der Treue.
Klingt an und hebt die Gläser hoch,
die alten Burschen leben noch,
noch lebt die alte Treue.
\endverse
\endsong


