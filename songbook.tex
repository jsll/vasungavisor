% Vasa nations sångbok Vasungavisor 2010

\documentclass[a6paper,8pt,makeidx]{book}
\usepackage[a6paper]{geometry}
\usepackage[T1]{fontenc}
\usepackage[utf8]{inputenc} %UTF8 stöd så att jag kan skriva åäö
\usepackage[swedish]{babel}
\usepackage{fancyhdr}
\usepackage[pdftex]{graphicx}	% Kuvat kyttn, tukee .png, .jpg ja .pdf-kuvia
\usepackage{import}		% Verbatim-tekstityyppi

\usepackage[lyric]{songs}

% Sidnummer samt header/footer
\pagestyle{empty}

% Antal kolumner
\songcolumns{1}

% Tjockleken på skiljelinjen mellan sångerna
\setlength{\sbarheight}{0pt}

% Tjockleken på linjen framför refrängen
\setlength{\cbarwidth}{1pt}

% Storleken på texten
\renewcommand{\lyricfont}{\footnotesize}

% Storleken på rubrikerna
\renewcommand{\stitlefont}{\sffamily\large}

% Hindrar sångerna från att stretcha över hela sidan
\renewcommand{\colbotglue}{0pt plus .5\textheight minus 0pt}

% Bakgrundsfärgen på låtnumrering och sångkommentarer
\definecolor{bgshade}{RGB}{235,235,235}
\renewcommand{\snumbgcolor}{bgshade} % låtnumrering
\renewcommand{\notebgcolor}{bgshade} % sångkommentarer

% Inga nummer på verserna
\noversenumbers

% Skapa register med namnet 'titleidx' för filerna
\newindex{titleidx}{titleidx}

\begin{document}

\begin{songs}{titleidx}

\import{./IHoganNord/}{nord.tex}
\clearpage

\import{./IUngdomensVar/}{ungdom.tex}
\clearpage

\import{./UrTorstigaStrupar/}{strupar.tex}	
\clearpage

\import{./SaHarligBouquet/}{bouquet.tex}	
\clearpage

\import{./ArrakAckJa/}{arrak.tex}	
\clearpage

\import{./FafaSkaSjung/}{fafa.tex}	
\clearpage

\import{./KunBotnialleSisaanAstuu/}{botnia.tex}	
\clearpage

\import{./AuldAcquaintance/}{auld.tex}	
\clearpage

\import{./BlomdoftIMelodin/}{blomdoft.tex}
\clearpage

\import{./Silliskapitel/}{sillis.tex}
\clearpage

\end{songs}

% Registrets utseende
\renewcommand{\idxtitlefont}{\rmfamily\mdseries}
\renewcommand{\idxlyricfont}{\rmfamily\mdseries}
\renewcommand{\idxheadfont}{\sffamily\it\large}
\setlength{\idxheadwidth}{0.5cm}
\showindex{Register}{titleidx}
\end{document}